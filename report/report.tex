\documentclass[11pt,a4paper]{report}

\usepackage{graphicx}
\graphicspath{ {images/} }

\author{Catherine Vlasov}
\title{3rd Year Project Report}
\date{May 2018}

\begin{document}

\makeatletter
	\begin{titlepage}
		\vspace*{\fill}
		\begin{center}
			{\huge \bfseries \@title }
			\\[4ex]
			{\LARGE  \@author}
			\\[2ex]
			{\large \@date}
			\\[50ex]
			\includegraphics[width=30mm]{oxlogo.png}
		\end{center}
		\vspace*{\fill}
	\end{titlepage}
\makeatother

\begin{abstract}
This is my abstract
\end{abstract}

\tableofcontents


%-----------------------
\chapter{Introduction}


%-----------------------
\chapter{Background}


\section{Reinforcement Learning}


\section{Learning Algorithms}


\subsection{Monte Carlo Learning}


%-----------------------------------------
\chapter{Design \& Implementation}


\section{Overview} % including a UML diagram


\section{Agents}


\section{Tic-Tac-Toe}

\subsection{Rules}

\subsection{Implementation Details}

\subsection{Breaking Symmetries}
% 1. Limiting Available Actions
% 2. Changing Definition of Equality


\section{Chung Toi}

\subsection{Rules} % including an example game board

\subsection{Implementation Details}

\subsection{Breaking Symmetries} % explanation of why I didn’t bother


%------------------
\chapter{Results}


\section{Overview}


\section{Monte Carlo Learning Parameter}
% - Tic-Tac-Toe win rate graph for each value of alpha from 0.01 to 0.99
% - zoomed in graph for the region around global max. with more games per value
% - same graphs for both implementations of symmetry-breaking
% - analysis


\section{Policies} % ???
% - pick interesting examples and plot the policies vs. number of games ???


\section{Time}
% - discussion of time complexity


\section{Win/Learning Rate}
% - plot all three Tic-Tac-Toe types on the same graph



%-------------------
\chapter{Analysis} % or Discussion



%-----------------------
\chapter{Conclusion}


\section{Future Work}


\section{Personal Reflections}

This has been a really great learning experience, not only in terms of discovering the field of reinforcement learning and its applications, but also in terms of gaining experience with building up a large, complex programming project from scratch using a variety of technical tools.

When I was in high school, I designed and implemented the classic board game Nine Men’s Morris where a user plays against an algorithm I wrote. My plan for the project was to ask my friends and family to play my game and to store the moves made in each game so that I could build up a database of sequences of moves and outcomes of the game. I hoped to somehow compute which moves were the best and get my algorithm to learn which moves made winning the most likely. Little did I know, I wanted to invent RL from scratch – unaware of its existence or of the amount of research in the field. Several years later, I am delighted to have had the chance to work on a similar project, but this time on a deeper level using well-known computational algorithms.

When I started working on this project, I decided to treat it as an opportunity to develop many skills that are critical for a career in software engineering:

\begin{itemize}
	\item[-] Writing clean, maintainable, well-documented code
	\item[-]Designing and implementing tests for my code
	\item[-]Using a build tool to manage dependencies between packages in my project as well as with external libraries
	\item[-]Using version control effectively
\end{itemize}

I tested the methods in my API using a unit-testing framework for Java called JUnit. I also used a testing framework for Java called Mockito – this allowed me to verify the behaviour of objects with external dependencies by creating “mock objects” for theses dependencies, which mimic real objects but do so in a particular way that I can specify.

In order to save my experiment results in a format that would facilitate the creation of graphs, I used opencsv, a CSV parser library for Java.

To manage my project’s dependencies, I used a build tool developed by Google called Bazel and I used Git for version control.

Overall, I really enjoyed learning about RL algorithms and exploring their applications and success rates, all while developing strong programming skills that will help me throughout my career.


%--------------------------------
\chapter{Acknowledgements}

Thanks for reading.


\end{document}